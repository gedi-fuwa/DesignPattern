\chapter{Design Patterns}

\section{Singleton}

\subsection{Verwendung}

Um das Singleton Design Pattern für eigene Klassen zu verwenden muss der Klasse, die als Singleton modelliert werden soll, lediglich der Stereotyp \textit{Singleton} hinzugefügt werden. Anschließend wird bei der Code-Generierung eine \textit{getInstance} Methode erstellt, mit welcher die einzelne Instanz der Klasse verwendet werden kann.

\subsection{Zu beachten}

Es dürfen keine Methoden vorhanden sein die den Namen getInstance besitzen.\\

\section{Observer}

\subsection{Verwendung}

Die Verwendung der Observer und Subject Klassen ist denkbar einfach: Hierzu muss
der Nutzer mindestens zwei Klassen im Klassendiagramm erstellen, denen dann die
Stereotypen Observer und Subject zugewiesen werden. Zwischen dem Observer und
den Subjects muss eine Assoziation modelliert werden. \\
Die update()-Funktionen der Observer muss selbst geschrieben werden, damit die
Observer auch eine Aktion ausführen können, wenn das Subject die notify-Funktion ausführt.

\subsubsection{Beispiel}

\subsection{Zu beachten}

Es dürfen keine Klassen oder Methoden vorhanden sein, die bereits Observer oder
Subject bzw. update oder register_observer delete_observer heißen.\\

Falls nur eine der benötigten Stereotypen zugewiesen wurde, gibt Rhapsody eine
entsprechende Meldung aus, erlaubt aber die Simplification.\\

