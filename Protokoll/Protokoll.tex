
\documentclass[a4paper]{article}
\usepackage[paper=a4paper,
			left=30mm,
			right=25mm,
			top=25mm,
			bottom=25mm]{geometry}
\usepackage[utf8]{inputenc}
\begin{document}
%############################## Titel ##############################
\begin{center}
{\Huge Protokoll}\\
{\large Semesterprojekt Designpattern mit IBM Rhapsody}\\
\end{center}
\vspace*{10mm}
%############################## Daten ##############################
\textbf{Datum:} 23.03.2015 \hfill \textbf{Protokollant:} Philipp Schneider\\
\\
\textbf{Anwesende:} Prof. Dr. R.Müller, Eduard Albach, Felix Penthin, Marco Haas, Eric Saier, Philipp Schneider\\
%############################## Inhalt ##############################
\section*{Protokollpunkte}
 \begin{itemize}
      \item Projekt- und Laboreinführung
      \item Planung weiterer Projekttreffen
      \begin{itemize}
         \item Wöchentliches Protokoll per Mail
         \item Ortswechsel: Sofaecke
      \end{itemize}
      \item Dokumentation wird in Latex geschrieben (WA Vorlage der HFU)
      \item Datenaustausch erfolgt über Git Repository
   \end{itemize}
%############################## Aufgaben ##############################
\section*{Aufgabenverteilung}
\begin{tabular}{|l|l|l|}\hline
   \textbf{Aufgabe} & \textbf{Personen} & \textbf{Zeit} \\ \hline \hline
   Git Repository anlegen & Eduard & bis nächsten Termin \\ \hline
   Latex Dokumentations-Vorlage in Repository einpflegen & Felix & bis nächsten Termin \\ \hline
   Doku Aufgabenstellungs Spezifikation formulieren & Marco, Eric, Philipp & bis nächsten Termin \\ \hline
   Dateipfad für Simplifications-Beispiel in Felix bereit stellen & Prof. Dr. R. Müller & nächste Tage \\ \hline
 \end{tabular}
\end{document}