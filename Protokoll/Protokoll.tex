
\documentclass[a4paper]{article}
\usepackage[paper=a4paper,
			left=30mm,
			right=25mm,
			top=25mm,
			bottom=25mm]{geometry}
\usepackage[utf8]{inputenc}
\begin{document}
%############################## Titel ##############################
\begin{center}
{\Huge Protokoll}\\
{\large Semesterprojekt Designpattern mit IBM Rhapsody}\\
\end{center}
\vspace*{10mm}
%############################## Daten ##############################
\textbf{Datum:} \today \hfill \textbf{Protokollant:} Marco Haas\\
\\
\textbf{Anwesende:} Prof. Dr. R.Müller, Eduard Albach, Felix Penthin, Marco Haas, Eric Saier, Philipp Schneider\\
%############################## Inhalt ##############################
\section*{Protokollpunkte}
 \begin{itemize}
      \item Observer wurde in Absprache mit Prof. Dr. Müller abgeschlossen, da es 
      		nicht möglich ist dieses Pattern erfolgreich fertig zu stellen.
      \item Umsetzung des Guarded Call Patterns gut vorangeschritten. Es ist schon 	ausführbar und die Simplifizierung funktioniert auch. Lediglich die Code-Formatierung und Roundtrip ist noch offen.
      \item Was soll beim Roundtrip passieren? Im Code wird eine neue Funktion implementiert, welche beim Roundtrip ins Modell übernommen wird. Jedoch wird die Simplifizierung dann Fehler werfen weil es die bestehenden Methoden/Attribute für (z.B. Singleton) schon gibt. Dies soll durch unsere "Roundtrip-Implementierung" verhindert werden.
      \item Fehlerbehandlung ist durch eine Error-Klasse abgeschlossen.
      \item Tests bei Guarded Call:
      		\begin{itemize}
      			\item Multi-Threading soll über Papieranalyse getestet werden
      			\item Single-Threading ist "einfach" über Rhapsody Testbar (ob alle Funktionen durchlaufen werden)
      		\end{itemize}
      \item Die Projektergebnisse sollen am Ende in gezippter Form abgegeben werden.
   \end{itemize}
%############################## Aufgaben ##############################
\section*{Aufgabenverteilung}
\begin{tabular}{|l|l|l|}\hline
   \textbf{Aufgabe} & \textbf{Personen} & \textbf{Zeit} \\ \hline \hline
   Roundtrip-Klasse bereitstellen & Felix, Marco & 15.06.15 \\ \hline
   Singleton abschließen & Felix, Marco & 15.16.15 \\ \hline
   Guarded Call implementieren & Eric, Philipp & 15.16.15 \\ \hline
   Guarded Call Tests in Rhapsody & Felix, Marco, Eduard & 22.06.15 \\ \hline
   Dokumentation Singleton abschließen & Felix, Marco & 15.06.15 \\ \hline
   Dokumentation Observer abschließen & Marco, Eduard & 15.06.15 \\ \hline
   Dokumentation Guarded Call abschließen & Philipp, Eric, Eduard & 22.06.15 \\ \hline
   Präsentation Planen & - &  15.06.15 \\ \hline.
   Installationsbeschreibung lesen & Prof. Dr. Müller  & 15.06.15 \\ \hline
   Benutzerhandbuch, Guarded Call & - & offen \\ \hline
   Benutzerhandbuch, Allgemein Fehlerbehandlung & - & offen \\ \hline
 \end{tabular}
\end{document}