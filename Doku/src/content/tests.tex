\chapter{Tests}

In diesem Kapitel werden zu den drei Design Patterns die jeweiligen Testfälle erläutert.

\section{Singleton}

Um sicherzustellen, dass die Singleton-Klasse richtig implementiert worden ist, müssen neben dem Behandeln der Compiler-Fehler und Warnungen folgende Testfälle manuell überprüft werden: 

\begin{description}
  \item[1.] \hfill \\
  Als allererstes muss überprüft werden, ob der Standard-Konstruktor der Singleton-Klasse aufrufbar ist. Sollte dies der Fall sein, scheitert der Test, denn der Konstruktor sollte von außerhalb nicht aufrufbar, d.h. private sein.
  \item[2.] \hfill \\
  Weiterhin muss überprüft werden, ob man von einem bestehenden Singleton-Objekt eine Kopie erzeugen kann. Ist dies möglich, scheitert dieser Test, denn ein Singleton-Objekt darf nicht kopiert werden. Der Copy-Konstruktor muss auch private sein.
  \item[3.] \hfill \\
  Ein weiterer Test ist, dass geprüft werden muss, ob es möglich ist, ein Singleton-Objekt zur Laufzeit zu zerstören. Das Singelton-Objekt wird erst am Ende der Programmlaufzeit freigegeben. Kann das Objekt schon vorher zerstört werden, schlägt der Test fehl. Der Destruktor muss auch als private deklariert sein.
  \item[4.] \hfill \\
  Möchte man von außerhalb ein Objekt der Singleton-Klasse implementieren, muss dies über die einzige öffentliche Methode der Singleton-Klasse "GetInstance()" passieren. Hierbei wird ein neues Objekt angelegt, sofern noch keins vorhanden war, ansonsten wird einfach das "alte" Objekt zurückgegeben.
  \item[5.] \hfill \\
  Hat der Benutzer selbst eine GetInstance-Methode implementiert, muss Rhapsody bei der Erzeugung des Projektes einen Fehler ausgeben und die Erzeugung abbrechen.
  \item[6.] \hfill \\
  Nachdem das Singelton-Objekt mithilfe der getInstance()-Methode erzeugt wurde, liefert diese Methode stets die gleiche Instanz der Singleton-Klasse zurück. 
   
   
\end{description}

\section{Observer}

\section{Guarded Call}


