\chapter{Tests}

In diesem Kapitel werden zu den drei Design Pattern die jeweiligen Testfälle erläutert.

\section{Singleton}

Um sicherzustellen dass die Singleton-Klasse richtig implementiert ist, müssen folgende Fälle überprüft werden: 

\begin{description}
  \item[1.] \hfill \\
  Als allererstes muss natürlich überprüft werden ob der Standart-Konstruktor der 	Singleton-Klasse aufrufbar ist. Sollte dies der Fall sein scheitert der Test, denn der Konstruktor sollte von außerhalb nicht aufrufbar, d.h. private sein.
  \item[2.] \hfill \\
  Ebenfalls überprüft werden muss, ob man von einem bestehenden Singleton-Objekt eine Kopie erzeugen kann. Ist dies möglich scheitert auch dieser Test, denn ein Singleton-Objekt darf nicht kopiert werden, der Copy-Konstruktor muss auch private sein.
  \item[3.] \hfill \\
  Noch ein Fehler des Testes soll angezeigt werden wenn das Objekt zur Laufzeit zerstört werden will. Der Destruktor muss auch als private deklariert sein.
  \item[4.] \hfill \\
  Möchte man von außerhalb ein Objekt der Singleton-Klasse implementieren, muss dies über die einzige öffentliche Methode der Singleton-Klasse "GetInstance()" passieren. Hierbei wird ein neues Objekt angelegt, sofern noch keins vorhanden war, ansonsten wird einfach das "alte" Objekt zurückgegeben.
  \item[5.] \hfill \\
  Hat der Benutzer selbst eine GetInstance-Methode implementiert muss überprüft werden ob diese richtig implementiert wurde und auch nur ein Objekt erzeugt, und nach dem zweiten oder dritten Aufruf nur das erste Objekt zurückgibt. Ist dies nicht der Fall schlägt der Test fehl.
\end{description}

\section{Observer}

\section{Guarded Call}


