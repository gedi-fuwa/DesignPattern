\chapter{Implementierung}

In diesem Kapitel befindet sich der Quelltext des jeweiligen Pattern. Damit wird
nicht der Quelltext der Pattern gemeint, sondern die Umsetzung unserer
Simplification. Um eine eigene Simplification zu schreiben und auf die
Funktionen die Rhapsody bietet zuzugreifen, müssen einige Einstellungen
beachtet werden. Diese werden in der Dokumentation des Wintersemesters
erläutert. Da wir jedoch einige Anpassungen an den Pattern vorgenommen haben
konnten wir die User-Simplification des letzten Semesters nicht verwenden. 

\section{Singleton}
\lstinputlisting
    [caption={User-Simplification Singleton}
       \label{lst:javaclass},
       captionpos=t,language=JAVA]
 {content/pictures/HFUSingeltonSimplifier.java} 

\section{Observer}
Bei der Implementierung von Observer verlief zuerst alles nach Plan. Nach einiger
Zeit hat sich jedoch schnell gezeigt das wir unsere Implementierung nicht so wie
gedacht und gewünscht umsetzen können. Nach etlichen Stunden Arbeit mussten wir
hier das Design Pattern Observer abbrechen bzw. schließen.\\
Da wir bei diesem Pattern zwei Klassen haben auf die die Simplifizierung
Einfluss nimmt, hatten wir hier große Probleme.\\
Anfangs haben wir eine falsche Denkweise verfolgt. Uns war nicht klar das die
von Rhapsody gegebene \textit{onPostSimplification}-Methode für jede Klasse ausgeführt
wird.
Dies machte unseren Code nur sehr unübersichtlich und wir haben versucht in einem Durchlauf
beide Klassen, also die Observer- und Subject-Klassen
anzupassen. Durch langes Testen sind wir jedoch dahinter gekommen und haben
einen anderen Ansatz verfolgt.\\
Der andere Ansatz war, die erzeugten Klassen nach dem
gesetzten Stereotypen zu durchsuchen und nach der entsprechenden Klasse zu modifizieren.
Dies hat soweit auch funktioniert. Das Problem hierbei war der Typ \textit{Observer}. Egal ob
wir die Observer-Objekte in eine Liste oder einen Vektor speichern, wir müssen der 
Liste bzw. dem Vektor einen Typen zuweisen (um zu bestimmen welche Objekte im Behälter
gespeichert werden sollen). Und genau hier ist das Problem.\\
Dieses Problem wollten wir durch das Erzeugen einer Oberklasse \textit{Observer} lösen.
Sieht nach einer einfachen Lösung aus, jedoch steckt der Teufel ja bekanntlich
im Detail. So war es auch in diesem Fall. Auch nach mehreren Stunden
Dokumentationen durchforsten und etwas experimentieren, haben wir keinen Weg
gefunden um eine Klasse zu erzeugen.\\
Natürlich bietet die API eine solche Funktion, nur leider gibt es keine Beispiele
oder nähere Erklärung. So waren unsere Versuche eine Klasse zu erzeugen nicht
von Erfolg gekrönt. Selbst wenn das Implementieren der Klasse funktioniert
hätte, haben wir noch das Problem der Vererbung. Diese sollte natürlich auch
durch die Simplifizierung gelöst werden. Auch das haben wir nicht aus den
gegebenen Dokumenten geschafft. Wir dachten uns dabei dass der User einfach eine
Klasse erzeugen muss mit dem Namen Observer. Jedoch bringt das recht wenig, wenn
man hier noch zusätzlich die Vererbung herstellen muss.\\
So musste der User immer mehr der Simplifizierung übernehmen und das ist nun
wirklich nicht Sinn und Zweck einer solchen \enquote{automatischen} Simplifizierung.\\
\\
Ein anderes Problem waren die ganzen Imports. Verschiedene Bibliotheken müssen
inkludiert werden um einen problemlosen Ablauf sicherzustellen. Auch das sollte
nach Möglichkeit direkt bei der Simplifizierung geschehen. Das hat jedoch auch
nicht funktioniert. Somit hatten wir das Problem dass der User mehr für die
Simplifizierung machen muss als die eigentliche Simplifizierung selbst. Somit
macht unsere Umsetzung hier wenig Sinn.\\
Hinzu kam dass wir am Observer-Pattern die von uns implementierte Fehlerbehandlung
testen und auch eine Roundtrip-Kontrolle implementieren sollten. Beides musste
von diesem Pattern abgezogen werden und am Singleton getestet werden. Weiter
unten im Text befindet sich eine ausführliche Beschreibung dazu.\\ \\
Nach Absprache mit dem Betreuer sind wir zu dem Entschluss gekommen, dass wir
dieses Design Pattern hier abschließen und mit dem Guarded Call weiter machen.
Der bisher erzeugte Quellcode wird jedoch auch abgegeben, sodass auch zukünftige
Gruppen von diesen Erkenntnissen profitieren können.
\lstinputlisting
    [caption={User-Simplification Observer}
       \label{lst:javaclass},
       captionpos=t,language=JAVA]
 {content/pictures/HFUSingeltonSimplifier.java} 


\section{Guarded Call}

\section{Fehlerbehandlung}

Besitzt eine Benutzerklasse ungültigen Code für das ausgewählte Design Pattern (bspw. eine 
bereits vorhandene, vom Benutzer erstellte GetInstance-Methode beim Singleton Design Pattern) 
sollte der Code-Generierungsprozess abgebrochen werden. Hierbei wird eine spezifische Error-Klasse aufgerufen, die eine Methode in die zu erzeugende Klasse implementiert, welche im Klartext die Fehlermeldung beinhaltet. Dies verursacht einen Syntax-Error im C++-Code. Erst wenn diese Fehlermeldung aus dem Code entfernt wurde, kann wieder korrekter Code generiert werden.

\section{Roundtrip}

Wird für ein Modell Code generiert, wird überprüft, ob die Attribute und Methoden des entsprechenden Design Patterns bereits vorhanden sind. Ist dies der Fall, wird geprüft, ob es sich um die durch einen früheren Simplifizierungsvorgang erstellten Attribute/Methoden handelt, oder ob diese vom Nutzer erstellt wurden. Dann wird die Fehlerbehandlungsroutine durchgeführt. \\
Wird festgestellt, dass die Attribute/Funktionen aus einer früheren Simplifizierung stammen, werden sie mit dem Stand aus der Design Pattern Spezifikation überschrieben.
