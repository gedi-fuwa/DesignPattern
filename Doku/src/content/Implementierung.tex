\chapter{Implementierung}

In diesem Kapitel befindet sich der Quelltext des jeweiligen Pattern. Damit wird
nicht der Quelltext der Pattern gemeint, sondern die Umsetzung unserer
Simplification. Um eine eigene Simplification zu schreiben und auf die
Funktionen die Rhapsody bietet zu zugreifen, müssen einige Einstellungen
beachtet werden. Diese werden in der Dokumentation des Wintersemesters
erläutert. Da wir jedoch einige Anpassungen an den Pattern vorgenommen haben
konnten wir die User-Simplification des letzten Semesters nicht verwenden. 

\section{Singleton}
\lstinputlisting
    [caption={User-Simplification Singelton}
       \label{lst:javaclass},
       captionpos=t,language=JAVA]
 {content/pictures/HFUSingeltonSimplifier.java} 


\section{Observer}


\section{Guarded Call}


\section{Ungültige Benutzerklasse}
Besitzt eine Benutzerklasse ungültigen Code für das ausgewählte Design Pattern (bspw. eine bereits vorhandene, vom Benutzer erstellte getInstance Methode beim Singleton Design Pattern) sollte der Code-Generierungsprozess abgebrochen werden. Dabei wird die \textit{TBD} Methode der Klasse AbstractSimplifier aufgerufen, bei welcher fehlerhafter C++ Code generiert wurde, wodurch wiederum Rhapsody gezwungen wird den Simplification-Prozess abzubrechen.